\documentclass[oneside]{book}
\usepackage[utf8]{inputenc}
\usepackage{authblk}
\usepackage{setspace} 
\usepackage{amsmath}
\usepackage{textcomp}
\usepackage{amssymb}
\usepackage{geometry}
\usepackage{amsthm}
\usepackage{runic}
\usepackage{mathtools}
\usepackage{graphicx}
\usepackage[breaklinks=true,a4paper=true,pagebackref=true]{hyperref}
\graphicspath{ {figures/} }
\geometry{
 a4paper,
 total={170mm,257mm},
 left=20mm,
 top=20mm,
}
\title{Libraria Algebrae}
\author{Liam Gardner}
\date{\today}
%\doublespacing
\newcommand\tab[1][1cm]{\hspace*{#1}}
\newcommand\nextline{\newline\tab}
\newcommand\nextquestion{\newline\newline}
\newcommand\soln{$\text{sol}^\text{n}\text{ }$}
\newcommand\fs{\mbox{\large $\mathrlap{f}s\,$}\,}
\newcommand\thm[2]{\section*{Theorem: #1}\label{sec:#2}\addcontentsline{toc}{section}{Theorem: #1}}
\newcommand\propn[2]{\section*{Proposition: #1}\label{sec:#2}\addcontentsline{toc}{section}{Proposition: #1}}
\renewcommand\mod[1]{\text{ (mod #1)}}


\begin{document}
\DeclarePairedDelimiter\abs{\lvert}{\rvert}

\maketitle
\tableofcontents
\chapter{Linear Diophantine Equations in $\mathbb{Z}^2$}
\tab
Note that for the general equation $ax+by$, we assume that $ab\neq0$, since if one is 0, then the equation is trivial. We wish to answer three fundamental problems
\begin{enumerate}
\item Does there exist an integer solution?
\item If the answer is yes, find an integer solution.
\item Can we find \textit{all} solutions?
\end{enumerate}
\tab
It is common for existential theorems (those that say solutions exist) to not give a means of how to find said solutions.
\subsection{Example}
Solve $506x + 391y = 23$. Notice that $\gcd(506,391)=23$. Thus, by b\'ezout's\ lemma, a solution exists. We can use the EEA (Extended Euclidean Algorithm) to find a solution to the equation. 
\newline
\begin{center}
\begin{tabular}{|c|c|c|c|}
\hline
$x$ & $y$ & $r$ & $q$ \\
\hline
\hline
1 & 0 & 506 & 0 \\
0 & 1 & 391 & 0 \\
1 & -1 & 115 & 1.0 \\
-3 & 4 & 46 & 3.0 \\
7 & -9 & 23 & 2.0 \\
-17 & 22 & 0 & 2.0 \\
\hline
\end{tabular}
\end{center}
\tab
Thus, we know that $(7,-9)$ is a solution. Now, we can subtract the equation $506x+391y=23$ with $506x_0 + 391y_0=23$, which gives $506(x-x_0) + 391(y-y_0) = 0$. Thus, we get $506(x-x_0) = -391(y-y_0)$. We can divide by the GCD of 506 and 391 to get the equation $22(x-x_0) = -17(y-y_0)$. Since the GCD is a common divisor to 506 and 391, we get that both $\frac{506}{23}$ and $\frac{391}{23}$ are integers and are coprime to each other. Now, since we know that $-17\lvert -17(y-y_0)$, and that $-17(y-y_0)=22(x-x_0)$, we get that $-17\lvert 22(x-x_0)$. Thus, by CAD, we get $17\lvert (x-x_0)$. Therefore, we find that $x-x_0 = 17n$ for some $n\in\mathbb{Z}$. Thus, a solution for $x$ can be found with $x_0 + 17n$, $\forall n\in\mathbb{Z}$.
\nextline
Following the same process, we get that $y=y_0+22n$. Therefore, all solutions to the Linear Diophantine Equation $506x + 391y = 23$ are given by the points $(7 + 17n, -9 - 22n)$.
\newline
\newline
Solve $506x + 391y = 24$.
\nextline
There are no \soln: Since $23=\gcd(506,391)$, we get that $23\lvert 506x+391y$ however, $23\nmid 24$, therefore, we've run into a contradiction.
\newline
\newline
Solve $506x + 391y = 46 = 2\cdot23$. We know that $506\cdot7 + 391\cdot(-9) = 23$, thus if we multiply both sides by two, we get that $506(7\cdot2) + 391(-9\cdot2) = 2\cdot23 = 46$. $506(14) + 391(-18) = 46$ gives the \soln (14,-18).

\thm{LDET Part 1}{ldeta}

Suppose $a,b,c\in\mathbb{Z}$ and $ab\neq0$ then $ax+by=c$ has a \soln in integers if and only if $\gcd(a,b)\lvert c$.
\subsection{Proof of forwards direction}
\tab
Assume $ax+by=c$ has an integer \soln $(x_0, y_0)$. Let $d=\gcd(a,b)$. Since $d\lvert a$ and $d\lvert b$, and since $c=ax_0+by_0$, then by Divsibility of Integer Combinations, $d\lvert c$.
\subsection{Proof of backwards direction}
\tab
Let $d=\gcd(a,b)$, and assume that $d\lvert c$, thus $c=kd$ \fs integer $k$. Then By B\'ezout's\ Lemma, we can find $x_0$ and $y_0$ $\in\mathbb{Z}$ such that $ax_0+by_0=d$, but then $k(ax_0+by_0)=kd$. Thus $a(kx_0) + b(ky_0) = c$
\subsection{Remark}
\tab
if $d\lvert c$ then the proof tells us how to find a \soln.
\begin{enumerate}
\item solve $ax+by=d$ using EEA to get $(x,y)=(x_0,y_0)$.
\item take $x=kx_0$ and $y=ky_0$, where $k=\frac{c}{d}$.
\end{enumerate}

\thm{LDET Part 2}{ldetb}

Suppose that $(x_0, y_0)$ is a particular solution to the LDE $ax+by=c$.
\nextline
Then the set of all \soln $\in\mathbb{Z}$ is given by the following set:
$$S = \left\{
(x,y) \Huge\mid x=x_0 + \frac{b\cdot n}{\gcd(a,b)},\,\,\, y=y_0 - \frac{a\cdot n}{\gcd(a,b)}
\right\}$$
\subsection{Proof}
\tab
Let $D$ be the set of all integer solutions to $ax+by=c$. I.E.
$$D = \left\{(x,y) \mid x,y,\in\mathbb{Z}, ax+by=c \right\}$$
\tab
This can be proven by showing $S \subseteq D$ and $D \subseteq S$
\newline $S\subseteq D:$
\nextline
Let $(x,y) \in S$, thus $x=x_0 + \frac{bn}{d}$ and $y=y_0-\frac{an}{d}$.
$$ax+by = a\left(x_0+\frac{bn}{d}\right) + b\left(y_0 - \frac{an}{d}\right)$$
$$= ax_0 + by_0 = c$$
\tab
since by definition, we know that $x_0$ and $y_0$ are a particular solution to the equation. Therefore, $S\subseteq D$
\newline $D\subseteq S:$
\nextline
Let $(x,y)\in D$, thus $x,y\in\mathbb{Z}$ and $ax+by=c$. Since, $(x_0,y_0)\in D$, we know that $ax_0 + by_0 = c$. We can subtract $ax_0+by_0=c$ from $ax + by = c$ to get $a(x-x_0) + b(y-y_0) = 0$. Dividing by the GCD of $a$ and $b$, we get that $\frac{a(x-x_0)}{d} = \frac{-b(y-y_0)}{d}$. Thus, since $\frac{b}{d}\lvert \frac{a}{d}(x-x_0)$ then by CAD, we get that $\frac{b}{d}\lvert (x-x_0)$, since $\frac{b}{d}$ and $\frac{a}{d}$ are coprime. Then we get $x-x_0=n\frac{b}{d}\Rightarrow x=x_0+\frac{bn}{d}$ \fs $n\in\mathbb{Z}$. Similarly, we get $y=y_0-\frac{an}{d}$. Therfore, $(x,y)\in S$.
\section{More Examples}
$12x+18y=13$.
\nextline
$\gcd(12,18) = 6$. Since $6\nmid13$ the equation has no \soln by \hyperref[sec:ldeta]{LDET1}
\newline
\newline
$14x-49y=28$
\nextline
$\gcd(14,-49) = 7$. Since $7\lvert 28$ the equation has solutions by \hyperref[sec:ldetb]{LDET2}
\nextline
Consider $14x-49y=7\Rightarrow 2x-7y=1$ which has \soln $(4,1)$. If we multiply $14(4) - 49(1) = 7$ by 4, we get that $14(x_0\cdot4) - 49(y_0\cdot4) = 7$, and from this we can get all \soln using LDET2.
\newline
\newline
\tab
Find all \soln to $15x+35=5$.
\newline
\begin{center}
\begin{tabular}{|c|c|c|c|}
\hline
$x$ & $y$ & $r$ & $q$ \\
\hline
\hline
1 & 0 & 15 & 0 \\
0 & 1 & 35 & 0 \\
1 & 0 & 15 & 0.0 \\
-2 & 1 & 5 & 2.0 \\
7 & -3 & 0 & 3.0 \\
\hline
\end{tabular}
\end{center}
\tab
Thus, we know that $\gcd(15,35)=5$, and $5\lvert 5$, there is a \soln. By EEA, we find $x=-2, y=1$ is a \soln. Then, we can find the general solution using
\hyperref[sec:ldetb]{LDET2}
 to be $x=-2+\frac{35n}{5}\Rightarrow 7n-2$, $y=1+\frac{15n}{5}\Rightarrow 1+3n$
\subsection{Geometric Understanding}
\tab
Graphing the line $15x+35y=5$ or $3x+7y=1$, we can rearrange for $y$ to get $y=\frac{-3}{7}x+\frac{1}{7}$. Thus, picking any lattice point, we can construct a triangle of length 7 and height 3 from that point to find the next lattice point.

\begin{figure}[h]
\centering
\includegraphics[width=0.5\textwidth]{l24f0}
\caption{Triangle formed from moving between two lattice points}
\end{figure}

\subsection{Nonnegative \soln}
\tab
The solutions to $15x+35y=5$ is given by $(-2+7n, 1-3n)$, these will be nonnegative if $x\geq0$ and $y\geq0$.
$$-2+7n\geq 0 \iff n\geq \frac{2}{7}$$
$$1-3n \geq 0 \iff n \leq \frac{1}{3}$$
Thus, since there are no integer solutions in the range $\frac{2}{7}\leq n \leq \frac{1}{3}$, as $n\in\mathbb{Z}$ there are no nonnegative solutions.
\subsection{Find all integer \soln to $15x+35y^2 = 5$}
\tab
Let $Y=y^2$, then, since we have the \soln to $15x+35Y=5$, given by $(-2+7n, 1-3n)$, all we have to do is see when $1-3n$ is a perfect square.
One way to solve this is to say let $z=y^2$ and solve $1-3n=z$. We can also solve this algebraically
\newline
\begin{center}
\begin{tabular}{c c c}
$\iff$ & $1-3n = y^2$ & \\
$\iff$ & $-3n=y^2-1$ & \\
$\iff$ & $3\lvert y^2-1$ & by euclid's lemma \\
$\iff$ & $3\lvert(y-1)(y+1)$ & \\
\end{tabular}
\end{center}
\tab
By Euclid's Lemma, we know either $3\lvert(y-1)$ or $3\lvert(y+1)$, though not both. Suppose $3\lvert(y-1) \iff y-1=3k\iff y=3k+1\, \fs k\in\mathbb{Z}$
Then, if $y=3k+1$, we get that $y^2=(1+3k)^2=1-3(-2k-3k^2)$. Let $n=(-2k-3k^2)$, then we get $y^2=1-3n$. Thus, if we take $x=-2+7n=-2+7(-2-3k^2)$ and $y=1-3k$, we get a perfect square \soln to the diophantine equation. 
\nextline
If $3\lvert y+1$, then we have $y=-1+3l\, \fs l \in \mathbb{Z}$. Then $y^2=(-1+3l)^2=1-3(2l+3l^2)$. Thus, $y=1-3l$ and $x=-2+7(2l+3l^2)$. Therfore, we can generate infinitely many perfect square solutions.
\chapter{Congruence and Modular Arithmetic}
\section{Clockwork Arithmetic Analogy}
\tab
Imagine a clock, we know that a clock has 12 spokes. If we look at it and see the hour hand at 2, and we know that 12 has already passed, then we also know that the clock really means it's 14. Thus, we can say $2\approx14$
\section{Definition}
\tab
$\forall a,b\in\mathbb{Z}$, we say that ``$a$ is congruent to $b$ mod(ulo) 12'' if $m\lvert(a-b)$.
\nextline
Notation: $a\equiv b\mod{m}$
\section{Examples}
\tab
$m=1$ then $a\equiv b\mod{1} \iff 1\lvert (a-b)$ which is true $\forall a,b\in\mathbb{Z}$
\nextline
$m=2$ then $a\equiv b\mod{2} \iff 2\lvert (a-b) \iff a-b$ is even $\iff a$ and $b$ are both even or both odd.
\nextline
$2\equiv-116\mod{2}$, however $3\not\equiv 10024\mod{2}$
\nextline
$14\equiv 2\mod{12} \iff 12\lvert(14-2)$
\nextline
$6\equiv 26\mod{10} \iff 10\lvert(6-26)$
\nextline
$6\not\equiv -26\mod{10} \implies 10\nmid(6+26)$
\propn{Congruence is an Equivalence Relation}{CER}
\begin{center}
\begin{tabular}{l|l}
$\forall m\in\mathbb{N}, \forall a,b,c,\in\mathbb{Z}$ & Congruence is \\
\hline
\hline
$a\equiv a\mod{m}$ & symmetric \\
if $a\equiv b\mod{m}$ and $b\equiv c\mod{m}$ then $a\equiv c\mod{m}$ & transitive \\
$a\equiv b \mod{m}\implies b\equiv a\mod{m}$ & Reflexive
\end{tabular}
\end{center}
\tab
Remark: Any relation $a\sim b$ that satisfies all above properties is called an equivalence relation.
\nextline
In calculus, we say two functions $f(x)\sim g(x)$ are an equivalence relation if $f^\prime(x) = g^\prime(x)$
\section{Recap}
\tab
Fix $m\in\mathbb{N}$, $\forall a,b\in\mathbb{Z}$
\nextline
\begin{tabular}{l c l}
$a\equiv b\mod{m}$ & $\iff$ & $m\lvert (a-b)$ \\
& $\iff$ & $a-b=mk\,\, \fs k\in\mathbb{Z}$ \\
& $\iff$ & $a=b+mk$
\end{tabular}
\newline
\propn{Arithmetic Rules of Congruence}{ARC}
Suppose $a\equiv a^\prime \mod{m}$ and $b\equiv b^\prime \mod{m}$ then,
\begin{enumerate}
\item $a+b\equiv a^\prime + b^\prime \mod{m}$
\item $a-b\equiv a^\prime - b^\prime \mod{m}$
\item $ab\equiv a^\prime b^\prime \mod{m}$
\end{enumerate}
\subsection{Examples}
$$2\equiv 9\mod{7}\land3\equiv 17\mod{7}\implies 2+3\equiv9+17\mod{7}\implies 5=26$$
\newline
$$56\cdot30 \mod{40}$$
$$56=16+40\equiv 16\mod{40}$$
$$30=-10\equiv \mod{40}$$
$$56\cdot30 \equiv 16\cdot(-10)\mod{40}$$
$$\equiv-160\mod{40}$$
$$\equiv-40\cdot4\mod{40}\equiv 0\mod{40}$$
\subsection{Proof of addition}
\tab
Since $a\equiv a^\prime \mod{m}$ and $b\equiv b^\prime \mod{m}$ then $m\lvert(a-a^\prime)$ and $m\lvert(b-b^\prime)$
\nextline
We have $(a+b)-(a^\prime-b^\prime)=a-a^\prime + b-b^\prime$, thus by DIC we get $m\lvert((a+b)-(a^\prime+b^\prime))$. Therefore $a+b\equiv a^\prime + b^\prime \mod{m}$
\subsection{Remark on Division}
\tab Care is needed with division
$ab\equiv ac\mod{m} \not\implies b\equiv c\mod{m}$ even if $a\not\equiv 0 \mod{m}$
\nextline
$$10\equiv 4\mod{6}$$
$$2\cdot5 \equiv 2\cdot2 \mod{6}$$
$$5\not\equiv 2\mod{6}$$
\propn{Congruent Division}{CD}
\tab
If $ab\equiv ac\mod{m}$ and $a$ is coprime to $m$, then $b\equiv c\mod{m}$
\subsection{Proof}
\tab
$$ab\equiv bc\mod{m}\iff m\lvert (ab-ac)\iff m\lvert a(b-c)$$
\nextline
Then by CAD we get $m\lvert(b-c)$ since $m$ and $a$ are coprime\nextline
$\implies b\equiv c \mod{m}$
\nextline
By applying the above proposition repeatedly; if $a_1\equiv a^\prime_1 \mod{m}, a_2\equiv a^\prime_2 \mod{m}, \cdots a_n\equiv a^\prime_n \mod{m}$, then we get the following result
\begin{enumerate}
\item $a_1+\cdots+a_n\equiv a^\prime_1 + \cdots a^\prime_n \mod{m}$
\item $a_1-\cdots-a_n\equiv a^\prime_1 - \cdots a^\prime_n \mod{m}$
\item $a_1\cdots a_n\equiv a^\prime_1 \cdots a^\prime_n \mod{m}$
\item (special case) $\forall q \in \mathbb{N}, a^q \equiv \left(a^\prime\right)^q \mod{m}$
\end{enumerate}
\subsection{More Examples}
Simplify $4^10 \mod{18}$\nextline
$4^{10} = \left(4^2\right)^5 = 16^5=(18-2)^5$ \nextline
$(18-2)^5\equiv-2^5\mod{18}$ \nextline
$\equiv-32\mod{18}$ \nextline
$-32+2\cdot 18\mod{18}$ \nextline
$4\mod{18}$
\newline
\newline
Is $3^9 + 62^{2020} - 20$ divisible by 7?
\nextline
Let $n=3^9+62^{2020}-20$. We know that $7\lvert n \iff 7\lvert(n-0) \iff n\equiv0\mod{7}$
\nextline
We can compute $3^9=\left(3^3\right)^3=27^3=(28-1)^3$ \nextline
$\equiv (-1)^3\mod{7}$\nextline
$\equiv -1\mod{7}$\nextline
We also know that $62^{2020} = (63-1)^{2020} \equiv (-1)^{2020}\mod{7} \equiv 1$ \nextline
$20=21-1\equiv -1\mod{7}$ \nextline
Using \hyperref[sec:ARC]{The arithmetic rules} we get that $n\equiv -1+1-(-1)\mod{7}\equiv1\mod{7}$ and thus $n$ is not divisible by 7.
\thm{Congruence and Remainders}{CR}
\subsection{Example}
\tab
What day of the week is it going to be a year from now?
\nextline
Since days cycle every 7, let's determine $365\mod{7}$. We know that $350=50\cdot7$ and thus \nextline $365 = 350 + 15$ \nextline
$=7\cdot50+14+1$ \nextline
$=7\cdot50+7\cdot2+1$\nextline
$=7(50+2)+1$\nextline
$\equiv 1\mod{7}$.\nextline
$365\equiv1\mod{7}$\nextline
Therefore, the day of the week one year from now is the same as the day of the week tomorrow.
\subsection{Observation}
\tab
if $n\in\mathbb{Z}$ \nextline
Any block of consecutive numbers will cycle through the numbers 0-6 inclusive:
$$\{\cdots,-7,-6,-5,-4,-3,-2,-1,0,1,2,3,4,5,6,7,8,9,10,11,12,13,14,\cdots\}$$
$$\equiv\{\cdots,0,1,2,3,4,5,6,0,1,2,3,4,5,6,0,1,2,3,4,5,6,\cdots\}\mod{7}$$
\section{Warning}
\tab
$\forall a,b,b^\prime \in \mathbb{N}$, if $b\equiv b^\prime \mod{m}$ then in general $a^b\not\equiv a^{b^\prime}\mod{m}$
\subsection{Example}
\tab
$4\equiv1\mod{3}$\nextline
$2^4=16\equiv1\mod{3}$ however $2^1\equiv 2\mod{3}$ \nextline
Thus $4\equiv1\mod{3}$ however $2^4\not\equiv2^1\mod{3}$
\propn{Finite Integers}{FI}
$\forall a,b,\in\mathbb{Z} a\equiv b \mod{m}\iff a$ and $b$ have the same remainder after division by $m$
\subsection{Proof}
\tab
Applying the division algorithm, we get $a=qm+b$ and $b=q^\prime m+r^\prime$ where $0\leq r,r^\prime < m$.\nextline
Notice if $a\equiv b\mod{m}$, we get that $m\lvert(a-b)$ and thus $m\lvert(qm+r - q^\prime m - r^\prime)$\nextline
$\implies m\lvert(m(q+q^\prime) + (r-r^\prime))$ Then by DIC it follows that \nextline
$\iff m\lvert (r-r^\prime)$ \nextline
Now since $0\leq r < m$ and $0 \leq r^\prime < m$, we get that $-m\leq r-r^\prime < m$. Now, since $m\lvert(r-r^\prime)$, we get that by BBD, that $m\leq \abs{r-r^\prime}$ and so for both inequalities to hold, $r=r^\prime$.
\end{document}